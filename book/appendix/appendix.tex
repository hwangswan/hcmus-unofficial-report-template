\chapter{Các định lý mở rộng}

\section{Một số định lý trong giải tích}

\subsection{Định lý giá trị trung bình}

\begin{theorem}[Định lý giá trị trung bình\cite{stewart12}]
Cho $f$ là một hàm liên tục trên khoảng đóng $[a, b]$ và khả vi trên khoảng mở $(a, b)$. Khi đó tồn tại một điểm $c \in (a, b)$ sao cho
$$
f'(c) = \frac{f(b) - f(a)}{b - a}
$$
\end{theorem}

\subsection{Định lý cơ bản của giải tích}

\begin{theorem}[Định lý cơ bản thứ I của giải tích\cite{stewart12}]
Cho $f$ là một hàm số thực liên tục trên khoảng đóng $[a, b]$. Gọi $F$ là hàm số xác định với mọi $x \in [a, b]$ sao cho
$$
F(x) = \int_a^x f(t)dt
$$
Khi đó $F$ liên tục trên $[a, b]$ và khả vi trên $(a, b)$, đồng thời với mọi $x \in (a, b)$, 
$$
F'(x) = f(x)
$$
vì thế $F$ là một \textbf{nguyên hàm} của $f$.
\end{theorem}

\begin{theorem}[Định lý cơ bản thứ II của giải tích\cite{stewart12}]
Cho $f$ là một hàm số thực trên khoảng đóng $[a, b]$ và $F$ là một hàm số liên tục trên $[a, b]$ và là một nguyên hàm của $f$:
$$
F'(x) = f(x)
$$
Nếu $f$ khả tích Riemann trên $[a, b]$ thì
$$
\int_b^a f(x)dx = F(b) - F(a)
$$
\end{theorem}