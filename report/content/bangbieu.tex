\section{Bảng biểu}
Bảng biểu được thể hiện như bảng~\ref{tab:my_label}, lưu ý flag \texttt{[H]} để disable floating (bảng được hiển thị đúng vị trí, không trôi lên đầu trang). Bảng~\ref{tab:my_label} là một trường hợp không sử dụng tag \texttt{[H]} và bảng bị trôi tít lên đầu trang:
\begin{table}%[H]
    \centering
    \begin{tabular}{lc}
        \toprule
        \textbf{Tên con vật} & \textbf{Số chân} \\ \midrule
        Gà                   & 2                \\ \midrule
        Chó                  & 4                \\ \midrule
        Trần Hoàng Tử        & 2                \\
        \bottomrule
    \end{tabular}
    \caption{Số chân của một số con vật, không có tag \texttt{[H]}}\label{tab:my_label}
\end{table}

Bảng~\ref{tab:my_label_with_H_tag} thể hiện bảng biểu với tag \texttt{[H]}\footnote{Tương tự cách sử dụng tag \texttt{[H]} với hình}. Để không phải mất thời gian tuổi trẻ ngồi chỉnh table, xài \href{https://www.tablesgenerator.com}{https://www.tablesgenerator.com}.

\begin{table}[H]
    \centering
    \begin{tabular}{l c}
        \toprule
        \textbf{Tên con vật} & \textbf{Số chân} \\ \midrule
        Gà                   & 2                \\ \midrule
        Chó                  & 4                \\ \midrule
        Trần Hoàng Tử        & 2                \\
        \bottomrule
    \end{tabular}
    \caption{Số chân của một số con vật, có tag \texttt{[H]}}\label{tab:my_label_with_H_tag}
\end{table}.